% ju 26-5-22 Lerntools.tex
\documentclass[a4paper,12pt,fleqn,parskip=half]{scrartcl}
\include{content/praeambel-artikel}   

\usepackage[left=2cm,right=2cm,top=1cm,bottom=1cm,includeheadfoot]{geometry}
%\usepackage[left=4cm,right=2cm,top=1cm, bottom=1cm,includeheadfoot]{geometry}
%\usepackage[landscape=true,left=2cm,right=2cm,top=1cm,bottom=1cm,includeheadfoot]{geometry}%quer

% eigene Farbe definieren
% Adobe Prozessfarben: CMYK: 100,50,0,35 -> 1,0.5,0,0.35
\definecolor{orange}{cmyk}{0,0.55,0.61,0}   % 0,55,61,0
\definecolor{blau5}{cmyk}{1,0.77,0.1,0.01}  % 100,77,10,
\definecolor{rot5}{cmyk}{0.22,1,1,0.19}     % 22,100,100,19
\definecolor{grau2}{cmyk}{0,0,0,0.1}        % 0,0,0,40
\definecolor{blau}{cmyk}{0.93,0.66,0,0.21}% 

% Literatur
\bibliography{content/literatur}
\bibliography{content/literatur-kfz}
\bibliography{content/literatur-sport}

%%%%%%%%%%%%%%%%%%%%%%%%%%%%%%%%%%%%%%%%%%%%%%%%%%%%%%%
\newcommand{\name}{Jan Unger}% anpassen!!!!!
\newcommand{\thema}{Lerntools}% anpassen!!!!!
\newcommand{\quelle}{\name}
\newcommand{\website}{https://bw-ju.de/}
\newcommand{\github}{https://github.com/ju1-eu}
%%%%%%%%%%%%%%%%%%%%%%%%%%%%%%%%%%%%%%%%%%%%%%%%%%%%%%%

\ihead{\textbf{Quelle:} \quelle}%{Kopfzeile innen}
\ohead{\textbf{Datum:} \today}  %{Kopfzeile außen}
\ifoot{\textbf{Thema:} \thema}  %{Fußzeile  innen}
\ofoot{Seite {\thepage} von {\pageref{LastPage}}}%{Fußzeile  außen}

\title{\thema}
\author{\name}
\date{\today}

\begin{document}
	%\thispagestyle{empty}
	%\maketitle
	%\newpage
	%\setcounter{page}{1}

	%%%%%%%%%%%%%%%%%%%%%%%%%%%%%%%%%%%%%%%%%%%
	\begin{center}
		\textbf{\Large \thema}\\%14pt
		\vspace{0.8em}
		%\datum	
		%\qrcode[hyperlink,level=Q,version=2,height=1cm]{\website}
		\qrcode[hyperlink,level=Q,version=2,height=1cm]{\github}
	\end{center}
	%%%%%%%%%%%%%%%%%%%%%%%%%%%%%%%%%%%%%%%%%%%

	\subsection*{Kompetenz = Wissen + Können}%\label{sec:Deadline}\index{Deadline}
	% Checkliste
	\begin{itemize} 
		\item [$\square$] \textbf{Verhältnis}: Wissen abrufen \& wiederholen (60:40) 
		\begin{itemize} 
			\item z. B. 1,5~h $\to$ (Ü) 1:00~h und (W) 30~Min.
		\end{itemize}
		\item [$\square$] \textbf{ABC-Liste} - Assoziatives Denken, (Ü) 3~Min., Durchgänge (25x)
		\begin{itemize} 
			\item Regeln: Kategorisieren (Was ist gleich?), mit den Augen wandern, mehr oder leer, Listen nummerieren und sammeln
			\item Vergleichen (Wissensaustausch), Fragen beantworten, Googeln $\to$ Erweitern des eigenen Wissensnetzes
		\end{itemize}
		\item [$\square$] \textbf{KaWa} - Wortbild (DIN-A4/A3) - Zusammenhang $\to$ >>Wissen verknüpfen, nicht isolieren!<<
		\item [$\square$] \textbf{Mäntylä-Liste} $\to$ Assoziieren zu Begriffen, später  Rekonstruieren der Begriffe
		\begin{itemize} 
			\item vor Lerneinheit, (Ü) 10~Min.
			\item Fokus auf das Wesentliche $\to$ Abfragen nach Stunden / Woche / Monat
		\end{itemize}
		\item [$\square$] \textbf{FLT} - Fragen mit Lückentext für Antwort generieren, (Ü) 25~Min. und (W) 15~Min.
		\item [$\square$] Fragen formulieren vor Lerneinheit $\to$ Geist öffnen, Aufmerksamkeit, Interesse wecken
		\begin{itemize} 
			\item Fragen ist der Schlüssel zum Begreifen / Merken
		\end{itemize}
		\item [$\square$] \textbf{Lernkarten} $\to$ Begriff erklären oder umschreiben um zu erraten, Kategorisieren
		\item [$\square$] \textbf{Hörtexte} (Rhetoriktraining)  $\to$ Aktiv \& passiv Hören $\to$ unbewusst Lernen (Inzidentales Lernen, nebenbei), Durchgänge (5x)
		\begin{itemize} 
			\item [\textcircled{1}] Vorlesen und Audio- oder Videoaufzeichnung
			\item [\textcircled{2}] vertiefendes Lesen (1. so schnell wie möglich + 2. halblaut)
			\item [\textcircled{3}] Mental - Training (lautlos Lesen)
			\item [\textcircled{4}] Vorlesen und Audio- oder Videoaufzeichnung (Unterschiede zur nächsten Übungsphase wahrnehmen)
		\end{itemize}
		\item [$\square$] Stegreifrede (Rhetoriktraining) $\to$ eine Minute zum Thema reden
		\item [$\square$] \textbf{Beamer} - Vortrag (Rhetoriktraining)
		\begin{itemize} 
			\item Durchgänge (1-15-25x) je mit Audio- oder Videoaufzeichnung
		\end{itemize}
	\end{itemize}

	%\input{content/tex/.tex}

    % Bibliographie
    \printbibliography[category=cited]
\end{document}

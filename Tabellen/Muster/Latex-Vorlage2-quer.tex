% ju 28-4-22 
\documentclass[a4paper,fontsize=14pt,DIV=calc,fleqn,parskip=half]{scrartcl}
\include{content/praeambel-artikel}  

%\usepackage[left=2cm,right=2cm,top=1cm,bottom=1cm,includeheadfoot]{geometry}
%\usepackage[left=4cm,right=2cm,top=1cm, bottom=1cm,includeheadfoot]{geometry}
%\usepackage[left=6cm,right=1cm,top=1cm, bottom=1cm,includeheadfoot]{geometry}
\usepackage[landscape=true,left=2cm,right=2cm,top=1cm,bottom=1cm,includeheadfoot]{geometry}%quer

% eigene Farbe definieren
% Adobe Prozessfarben: CMYK: 100,50,0,35 -> 1,0.5,0,0.35
\definecolor{orange}{cmyk}{0,0.55,0.61,0}   % 0,55,61,0
\definecolor{blau5}{cmyk}{1,0.77,0.1,0.01}  % 100,77,10,
\definecolor{rot5}{cmyk}{0.22,1,1,0.19}     % 22,100,100,19
\definecolor{grau2}{cmyk}{0,0,0,0.1}        % 0,0,0,40
\definecolor{blau}{cmyk}{0.93,0.66,0,0.21}% 

\addtokomafont{section}{\color{blau5}}

% Literatur
\bibliography{content/literatur}
\bibliography{content/literatur-kfz}
\bibliography{content/literatur-sport}

%%%%%%%%%%%%%%%%%%%%%%%%%%%%%%%%%%%%%%%%%%%%%%%%%%%%%%%
\newcommand{\name}{Jan Unger}% anpassen!!!!!
\newcommand{\thema}{\LaTeX -Vorlage}% anpassen!!!!!
\newcommand{\quelle}{\name}
\newcommand{\website}{https://bw-ju.de/}
\newcommand{\github}{https://github.com/ju1-eu}
%%%%%%%%%%%%%%%%%%%%%%%%%%%%%%%%%%%%%%%%%%%%%%%%%%%%%%%

%\ihead{\textbf{Quelle:} \quelle}%{Kopfzeile innen}
%\ohead{\textbf{Datum:} \today}  %{Kopfzeile außen}
\ifoot{\textbf{Thema:} \thema}  %{Fußzeile  innen}
\ofoot{Seite {\thepage} von {\pageref{LastPage}}}%{Fußzeile  außen}

\title{\color{blau5}{\thema}}
\author{\name}
\date{\today}

\begin{document}
	\thispagestyle{empty}
	\maketitle
	\newpage
	\setcounter{page}{1}

	\section*{Listen}%\label{sec:listen}\index{Listen}
	% Checkliste
	\begin{itemize}[label=\checkmark] %\itemsep -2pt
		\item Check
	\end{itemize}

	% Checkliste
	\begin{itemize} 
		\item [$\square$] Check
	\end{itemize}

	% Liste
	\begin{itemize} 
		\item Punkt
	\end{itemize}
	% Liste numeriert
	\begin{enumerate} 
		\item Liste numeriert
		\item Punkt
	\end{enumerate}

	\newpage
	\section*{Abbildungen}%\label{sec:Abbildungen}\index{Abbildungen}

	Liste

	\begin{itemize} 
		\item Punkt
		\item Punkt
		\item Punkt
		\item Punkt
	\end{itemize}
			
	\begin{figure}[!h]
	\centering
	\subcaptionbox{Logo 1 \label{logo1}}
	{\includegraphics[width=0.25\textwidth]{images/Logo/Logo1}}
	\subcaptionbox{Logo 2 \label{logo2}}
	{\includegraphics[width=0.25\textwidth]{images/Logo/Logo2}}
	\caption{Zwei Logos}\label{Logos}
	\end{figure}

	\newpage
	\section*{Abbildungen 2}%\label{sec:Abbildungen2}\index{Abbildungen2}
	%Logo in Neg, Grau, Schwarz (\autoref{fig:logoneggrauschwarz}).
	%
	\begin{figure}[!h]% hier: !hb
		\centering
		\begin{minipage}[b]{0.40\textwidth}
			\includegraphics[width=\textwidth]{images/Logo/Logo-negativ}%
		\end{minipage}
		\hfill
		\begin{minipage}[b]{0.30\textwidth}
			\includegraphics[width=\textwidth]{images/Logo/Logo-Grau}%
		\end{minipage}
		\hfill
		\begin{minipage}[b]{0.20\textwidth}
			\includegraphics[width=\textwidth]{images/Logo/Logo-SW}%
		\end{minipage}
		\caption{Logo in Neg, Grau, Schwarz}\label{fig:logoneggrauschwarz}%% anpassen
	\end{figure}

	\newpage
	\section*{Text und Abbildungen}%\label{sec:TAbbildungen}\index{Text, Abbildungen}
	%Logo in Neg, Grau, Schwarz (\autoref{fig:logoneggrauschwarz}).
	%
	\begin{figure}[!h]% hier: !hb
		\centering
		\begin{minipage}[c]{0.55\textwidth}
			Liste
			\begin{itemize} 
				\item Punkt
				\item Punkt
				\item Punkt
				\item Punkt
			\end{itemize}
		\end{minipage}
		\hfill
		\begin{minipage}[c]{0.35\textwidth}
			\includegraphics[width=\textwidth]{images/Logo/Logo2}%
		\end{minipage}
		%\caption{Logo in Neg, Grau, Schwarz}\label{fig:logoneggrauschwarz}%% anpassen
	\end{figure}

	\newpage
	\section*{Tabelle}%\label{sec:Tabelle}\index{Tabelle}

	\begin{table}[!h]% hier: !ht 
		\centering
		\caption{\textbf{Integer-Datentypen}}
		\label{tab:Integer-Datentypen}
		\begin{tabular}{@{}lll@{}}
		\toprule
		\textbf{Typ}       & \textbf{Wertebereich}    & \textbf{Speicherbedarf} \\ \midrule
		int                & –32768...32767           & 2 Bytes                 \\
		int                & –2147483648...2147483647 & 4 Bytes                 \\
		short int          & –32768...32767           & 2 Bytes                 \\
		unsigned short int & 0...65535                & 2 Bytes                 \\
		long int           & -2147483648...2147483647 & 4 Bytes                 \\
		unsigned long int  & 0...4294967295           & 4 Bytes                 \\ \bottomrule
		\end{tabular}%
	\end{table}

	%\input{content/tex/.tex}

    % Bibliographie
    \printbibliography[category=cited]
\end{document}
